\begin{Funções polinomiais}

\begin{Def}
  Sendo $(R,+,\cdot)$ um anel qualquer,
  dizemos que uma função
  $\funcao{f}{R}{R}$ é \Conceito{polinomial}{função polinomial}
  sobre o anel $R$ se e só se existe alguma função
  $\funcao{a}{\MMN}{R}$ de suporte finito para cujo polinômio $p$,
  para todo $x\in R$,
  \begin{equation*}
    f(x) = p(x)\MMp
  \end{equation*}
\end{Def}

\begin{Obs}
  Tal como definimos raízes para polinômios, definimos raízes para
  funções polinomiais: um elemento $u$ de $R$ é raiz de $f$ se e só se
  $f(u)=\zero$.
\end{Obs}

\begin{Lem}\label{lemxnun}
  Sendo $(R,+,\cdot)$ um anel e $n$ um número natural,
  \begin{equation*}
    x^n - y^n = (x-y)\sum_{k=0}^{n-1}y^kx^{n-k-1}\MMv
  \end{equation*}
  para quaisquer $x$ e $y$ elementos de $R$.
\end{Lem}

\begin{dem}
  Se $n=0$ então a igualdade se verifica trivialmente. Senão,
  é também imediato que
  \begin{equation*}
    \begin{aligned}
      (x-y)\sum_{k=0}^{n-1}y^kx^{n-k-1}
        &= \sum_{k=0}^{n-1}(y^kx^{n-k}-y^{k+1}x^{n-k-1}) \\
        &= \sum_{k=0}^{n-1}y^kx^{n-k} -
           \sum_{k=0}^{n-1}y^{k+1}x^{n-(k+1)}\\
        &= \sum_{k=0}^{n-1}y^kx^{n-k} -
           \sum_{k=1}^{n}y^{k}x^{n-k}\\
        &= y^0x^{n-0} + \sum_{k=1}^{n-1}(y^kx^{n-k} - y^{k}x^{n-k})
           - y^nx^{n-n}\\
        &= \um\cdot x^n - y^n\cdot \um\\
        &= x^n - y^n\MMv
    \end{aligned}
  \end{equation*}
  como queríamos demonstrar.
\end{dem}

\begin{Lem}
  Sendo uma função polinomial $f$
  sobre um domínio de integridade
  infinito $(R,+,\cdot)$ e sendo $p$ um polinômio de $R[x]$ tal que
  $p(x)=f(x)$, se $d(p)\neq 0$ e se $f$ possui uma raiz $u$ então
  \begin{equation*}
    f(x) = (x-u)q(x)\MMv
  \end{equation*}
  para alguma polinômio $q$ de $R[x]$ tal que
  $d(q) = d(f)-1$.
\end{Lem}

\begin{dem}
  Suponhamos que $f$ possua uma raiz $u$ e suponhamos também que
  $d(p)\neq 0$. Assim,
  sabemos que, para todo $x\in R$,
  $f(x)=f(x)-f(u)$, já que $f(u)=\zero$. Portanto, sendo $a$ a função de
  suporte finito de $p$,
  \begin{equation*}
    \begin{aligned}
      f(x) &= \sum_{k=0}^{d(p)} a_kx^k
              - \biggl(\sum_{k=0}^{d(p)} a_ku^k\biggr) \\
           &= \sum_{k=0}^{d(p)} a_k(x^k-u^k)\\
           &= \sum_{k=1}^{d(p)} a_k(x^k-u^k)
    \end{aligned}
  \end{equation*}
  e, do lema\xspace\ref{lemxnun},
  \begin{equation*}
    \begin{aligned}
      f(x) &= \sum_{k=1}^{d(f)} a_k\biggl(
              (x-u)\sum_{j=0}^{k-1}u^jx^{k-j-1}
              \biggr)\\
           &= (x-u)\Biggl(\sum_{k=1}^{d(f)}
                \biggl(\sum_{j=k}^{d(f)}a_ju^{j-1}\biggr)
                x^{k-1}\Biggr)\MMp
    \end{aligned}
  \end{equation*}
  No entanto, notemos que
  \begin{equation*}
    \Biggl(\sum_{k=1}^{d(f)}
                \biggl(\sum_{j=k}^{d(f)}a_ju^{j-1}\biggr)
                x^{k-1}\Biggr)
  \end{equation*}
  é uma função polinomial de grau $d(f)-1$ para a função de suporte
  finito construída pelos termos de $\sum_{j=k}^{d(f)}a_ju^{j-1}$,
  o que conclui nossa demonstração.
\end{dem}

\begin{Obs}\label{obsdecompfx}
  Indutivamente, é fácil perceber que,
  sendo $m$ um número inteiro positivo,
  sendo $a$ uma função de suporte finito para uma função polinomial $f$
  com raízes sobre um domínio de integridade
  infinito $(R,+,\cdot)$ tal que, para $a$,
  $d(f)\geq m$, e sendo $u$ uma injeção de $[m]$ em $\rho(f)$,
  existe uma função polinomial $q_m$,
  de grau $d(f)-m$ para alguma função de suporte finito, tal que
  \begin{equation*}
    f(x) = \biggl(\prod_{k=1}^m \bigl(x-u(k)\bigr)\biggr)q_m(x)\MMp
  \end{equation*}
\end{Obs}

\begin{Lem}\label{lemdipoli}
  Sendo $f$, $g$ e $h$ funções polinomiais sobre um domínio de
  integridade infinito $(R,+,\cdot)$, se um elemento $u$ de $R$ é uma
  raiz de $f$ e se $f=gh$ então $u$ é raiz de $g$ ou raiz de $h$.
\end{Lem}

\begin{dem}
  Suponhamos que $u$ seja raiz de $f$ e que $f=gh$. Suponhamos também
  que $u$ não seja raiz nem de $g$ nem de $h$. Assim,
  \begin{equation*}
    f(u) = (gh)(u) = g(u)h(u) = 0\MMp
  \end{equation*}
  Como $g(u)$ e $h(u)$ são elementos de $R$ e $(R,+,\cdot)$ é um domínio
  de integridade, $g(u)=0$ ou $h(u)=0$, o que é um absurdo, dado que $u$
  não é raiz nem de $g$ nem de $h$.
\end{dem}

\begin{Lem}\label{lemmaxdf}
  Sendo $a$ uma função de suporte finito para uma função polinomial $f$
  sobre um domínio de integridade
  infinito $(R,+,\cdot)$ tal que, para $a$,
  $d(f)\neq 0$, $f$ tem no máximo $d(f)$ raízes em $R$.
\end{Lem}

\begin{dem}
  Se $f$ não tem raízes então é imediato que
  $\cardi{\rho(f)}=0\leq d(f)$.
  Senão, podemos tomar um inteiro positivo conveniente $m$ e
  uma injeção $u$ de $[m]$ em $\rho(f)$. Portanto, da
  observação\xspace\ref{obsdecompfx}, existe uma função polinomial
  $q_m$,
  de grau $d(f)-m$ para alguma função de suporte finito, tal que
  \begin{equation*}
    f(x) = \biggl(\prod_{k=1}^m \bigl(x-u(k)\bigr)\biggr)q_m(x)\MMp
  \end{equation*}
  Logo, do lema\xspace\ref{lemdipoli},
  \begin{equation*}
    \rho(f) = u([m])\cup \rho(q_m)\MMv
  \end{equation*}
  e, conseqüentemente,
  \begin{equation*}
    \cardi{\rho(f)} \leq \cardi{u([m])} + \cardi{\rho(q_m)}
     = m + \cardi{\rho(q_m)}\MMp
  \end{equation*}
  Assim, se $\cardi{\rho(f)}>d(f)$ então
  \begin{equation*}
    \cardi{\rho(q_m)} > d(f)-m\MMv
  \end{equation*}
  o que contraria o que já concluímos. Dessarte,
  $\cardi{\rho(f)}\leq d(f)$, como queríamos demonstrar.
\end{dem}

\begin{Lem}\label{lemidentidadenula}
  Se $f$ é uma função polinomial identicamente nula sobre um domínio de
  integridade infinito $(R,+,\cdot)$ então para
  toda função de suporte finito
  $a$ para $f$ vale que $a(\MMN)=\cj{0}$.
\end{Lem}

\begin{dem}
  Suponhamos que haja alguma função de suporte finito
  $a$ para $f$ tal que seja verdade que $a(\MMN)\neq \cj{0}$. Assim, o
  conjunto $a^{-1}(R\setminus\cj{0})$ não é vazio e é finito,
  já que $a$ possui
  suporte finito. Portanto, existe um inteiro positivo $d$ tal
  que $d=d(f)$ para $a$. Assim, do lema\xspace\ref{lemmaxdf}, $f$ tem no
  máximo $d$ raízes, o que é um absurdo, já que, da
  nomenclatura\xspace\ref{nomraiz}, todo elemento de $R$, um conjunto
  infinito, é raiz de $f$.
\end{dem}

\begin{Teo}[Princípio da identidade de funções polinomiais]\label{teopip}
  \index{princípio da identidade de funções polinomiais}
  \mbox{}\marginpar%
  [\raggedright\hspace{0pt}\footnotesize %
    princípio da identidade de funções polinomiais\\%
    \vspace{\baselineskip}]%
  {\raggedleft\hspace{0pt}\footnotesize %
    princípio da identidade de funções polinomiais\\%
    \vspace{\baselineskip}}
  Sendo $f$ e $g$ duas funções polinomiais sobre um domínio de
  integridade infinito $(R,+,\cdot)$,
  sendo $a$ e $b$ funções de suporte finito para, respectivamente, $f$
  e $g$, e sendo $d(f)=n$ para $a$ e $d(g)=m$ para $b$,
  $f=g$ se e só se $a=b$.
\end{Teo}

\begin{dem}
  É imediato que se $n=m$ e se $a_k=b_k$ para cada
  $k\in[0..\max\cj{n,m}]$ então $f=g$. Para demonstrarmos a recíproca,
  suponhamos
  que $f=g$. Sabemos que
  \begin{equation*}
    (f-g)(x) = \sum_{k=0}^{\min\cj{n,m}} (a_k-b_k)x^k
             + \sum_{k=m}^n (a_k-b_k)x^k + \sum_{k=n}^m (a_k-b_k)x^k
  \end{equation*}
  e, como $f=g$, que $(f-g)(x)=f(x)-g(x)=\zero$. Portanto,
  \begin{equation*}
    \sum_{k=0}^{\min\cj{n,m}} (a_k-b_k)x^k
             + \sum_{k=m}^n (a_k-b_k)x^k + \sum_{k=n}^m (a_k-b_k)x^k
             = \zero\MMp
  \end{equation*}
  Porém, como temos do lema\xspace\ref{lemidentidadenula} que, para todo
  natural $k$, $a_k-b_k=0$, concluímos facilmente
  que $a_k=b_k$ e que $n=m$.
\end{dem}

\begin{Cor}
  Sendo $f$ uma função polinomial sobre um domínio de integridade
  infinito $(R,+,\cdot)$, a existência de uma função de suporte finito
  para $f$ é única, e, portanto, é único também o grau de $f$, se
  estiver definido.
\end{Cor}

\begin{dem}
  Sejas duas funções $a$ e $b$ de suporte finito para $f$. Do
  teorema\xspace\ref{teopip}, como $f=f$, temos que $a=b$, o que nos
  leva também à unicidade de $d(f)$.
\end{dem}

