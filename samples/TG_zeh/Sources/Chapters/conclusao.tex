%%%%%%%%%%%%%%%%%%%%%%%%%%%%%%%%%%%%%%%%%%%%%%%%%%%%%%%%%%%%%%%%%%%%%%%%
% conclusao.tex                                                        %
%                                                    início do arquivo %
%%%%%%%%%%%%%%%%%%%%%%%%%%%%%%%%%%%%%%%%%%%%%%%%%%%%%%%%%%%%%%%%%%%%%%%%

\chapter{Conclusão}

O conceito de irredutibilidade polinomial sobre corpos finitos vem sido
estudado profundamente nas últimas décadas, de cujos estudos derivaram
aplicações diversas, como aplicações na Combinatória, na Teoria de
Codificação, na
Criptologia, na Criptografia, no estudo de circuitos com
retroalimentação, na geração de seqüências pseudo-aleatórias, na Teoria
dos Números e na manipulação de símbolos algébricos. No presente
trabalho, após havermos construído um esboço teórico que fundamentou
nossa abordagem, apresentamos um algoritmo para testar a
irredutibilidade de um polinômio sobre um corpo finito, elaborado por
M. Rabin\cite{pralgffrabin}. Ademais,
ainda analisamos o custo computacional desse algoritmo e concluímos que
se trata de um algoritmo bastante eficiente. Para testar se um polinômio
de grau $n$ sobre um corpo finito com $p$ elementos é irredutível,
 mostramos que
o tempo do
algoritmo é dado por:
\begin{equation*}
  T(n) = O\bigl((n\log{n}(\log{\log{n}})\log{p})^2\bigr)\MMp
\end{equation*}
Além disso, ainda fornecemos o algoritmo num formato bastante passível
de implementação e indicamos uma bibliografia que contivesse
 sugestões de
im\-ple\-men\-ta\-ções eficientes para as operações utilizadas.

%%%%%%%%%%%%%%%%%%%%%%%%%%%%%%%%%%%%%%%%%%%%%%%%%%%%%%%%%%%%%%%%%%%%%%%%
% conclusao.tex                                                        %
%                                                       fim do arquivo %
%%%%%%%%%%%%%%%%%%%%%%%%%%%%%%%%%%%%%%%%%%%%%%%%%%%%%%%%%%%%%%%%%%%%%%%%


