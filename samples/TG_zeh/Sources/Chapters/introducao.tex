%%%%%%%%%%%%%%%%%%%%%%%%%%%%%%%%%%%%%%%%%%%%%%%%%%%%%%%%%%%%%%%%%%%%%%%%
% introducao.tex                                                       %
%                                                    início do arquivo %
%%%%%%%%%%%%%%%%%%%%%%%%%%%%%%%%%%%%%%%%%%%%%%%%%%%%%%%%%%%%%%%%%%%%%%%%

\chapter{Introdução}

Embora muitas das disciplinas da Matemática terem sido axiomatizadas e
organizadas logicamente há muitos séculos, e isso pode se verificar
quando tomamos como exemplo a Geometria, cuja primeira
a\-xi\-o\-ma\-ti\-za\-ção foi proposta em cerca de 300 a.C.
por Euclides em sua obra ``Elementos'', a
Álgebra só foi receber tal devido tratamento bastante tarde,
já que os
primeiros esforços nesse sentido datam no mínimo do século XIX, um
século bastante importante na história da Matemática Moderna. Se o
leitor se lembrar, foi justamente nesse século que Georg Cantor propôs
a Teoria (Ingênua) dos Conjuntos.

Porém, os conjuntos de Cantor não constituíram a única formulação
teórica importante do século. Em seu estudo sobre a resolução de
equações por radicais, grande febre da época,
o matemático francês Évariste Galois (1811--1832) propôs em 1830, pela
primeira
vez na história, o conceito de ``grupo'' com esse
nome. Vale ressaltar que isso não significa que o conceito de ``grupo''
não tivesse aparecido intuitivamente
antes, por exemplo, na obra de Joseph-Louis
Lagrange (1736--1813). No entanto, foi Galois quem lhe deu a devida
organização lógica e a\-xi\-o\-ma\-ti\-za\-ção, que utilizamos até hoje.

Assim como os grupos, outras estruturas algébricas aparecerem em meados
do século XIX. O conceito de ``corpo'', por exemplo, já aparecia
intuitivamente nas obras do norueguês Niels Henrik Abel (1802--1829) e
Galois
e
foi apresentado explicitamente pelos ``corpos de grau finito'' do alemão
R. Dedekind (1831--1916) em seu estudo sobre os ``números
algébricos''. A definição de ``anel'', por sua vez, embora tenha sido
formalizada apenas em 1914 pelo alemão A. Fraenkel (1891--1965), já
aparecia, inclusive com esse nome, nas obras de D. Hilbert
(1852--1943).

Dentro da história da Álgebra Moderna,
os corpos finitos, mais especificamente os corpos de Galois, são em
teoria tão ``velhos'' quanto a Teoria de Corpos propriamente
dita. Entretanto, foi apenas nas últimas décadas, com a emergência da
Matemática Discreta como uma disciplina que recebesse grande atenção dos
matemáticos, que os corpos finitos vieram ser estudados com mais
profundidade e interesse. Nesse ínterim, a Teoria de Corpos Finitos
acabou produzindo importantíssimas aplicações tanto na Matemática quanto
na Engenharia e na Tecnologia. Algumas das aplicações encontram-se
principalmente na Combinatória, na Teoria de Codificação, na
Criptologia, na Criptografia, no estudo de circuitos com
retroalimentação, na geração de seqüências pseudo-aleatórias, na Teoria
dos Números e na manipulação de símbolos algébricos.

Se conceitos como ``grupos'', ``corpos'' e ``anéis'' são relativamente
novos (ao menos explicitamente), o mesmo não ocorre com os
polinômios. Pode-se dizer que o
conceito de ``polinômio'' seja, ao menos intuitivamente,
tão antigo quanto a própria Álgebra e que remonte suas origens às obras
clássicas
gregas, das quais advieram
as três disciplinas ``Geometria'', ``Aritmética''
e ``Álgebra''. Contudo, foi apenas no século XVI, com a criação do
``Cálculo Literal'' por François Viète (1540--1603), que a linguagem das
fórmulas revolucionou a Matemática, tornando-se possível
generalizar expressões.
Todavia, foi pouco tempo depois, na obra de René Descartes (1596--1650),
que as expressões matemáticas ganharam a forma que utilizamos até
hoje. Foi Descartes que introduziu a convenção de se denotar variáveis
por $x$, $y$ e $z$ e constantes ou parâmetros por $a$, $b$ e $c$. Também
foi ele quem criou a notação exponencial para indicar potências e quem
pela primeira vez, muito provavelmente, usou o
princípio da identidade de polinômios. O mais
curioso, no entanto, é que a principal preocupação intelectual de
Descartes não era a Matemática, mas a Filosofia. Muitos hoje acreditam
que foi justamente por isso que Descartes publicou apenas um trabalho
matemático: sua obra entitulada ``\textit{Géometrie}'', listada
atualmente como uma das obras mais importantes de toda a história da
Matemática. Com tudo isso, podemos sem medo afirmar que o século XVII
foi um dos mais importantes para a história da Teoria de
Polinômios, mas não podemos negligenciar o fato de que conceitos
algébricos mais sutis, como o de irredutibilidade polinomial, só
receberam atenção juntamente com as transformações pelas quais a
Matemática passou no século XIX.

No presente trabalho, pretendemos introduz o leitor aos conceitos
elementares relacionados à irredutibilidade polinomial em corpos finitos
e ainda
apresentar um algoritmo utilizado para testar se um polinômio sobre um
corpo finito é ou não
irredutível. O algoritmo que mostraremos foi proposto por Michael
O. Rabin em 1978\cite{pralgffrabin}, com custo inferior aos custos dos
algoritmos publicados anteriormente. Para podermos, portanto,
abordar
os conceitos e o algoritmo, exporemos também um pouco da fundamentação
teórica necessária para as demonstrações utilizadas.

Assim, no capítulo\xspace\ref{capfund} nos dedicaremos principalmente a
construir um esboço de alguns conceitos da Teoria de Polinômios,
incluindo a irredutibilidade polinomial em corpos finitos. É muito
importante destacar que o esboço se atém especialmente a muito daquilo
que vai
ser utilizado posteriormente e que não contempla nem mesmo os resultados
mais interessantes ou de maior impacto, por fugiram do escopo da
pesquisa. O mesmo vale para as
demonstrações. Como alguns teoremas exigem conceitos mais avançados que
os que nós pretendemos abordar, dada a realidade deste
texto,
dedicar-nos-mos a fundamentar os conceitos mais
elementares e não nos preocuparemos tanto em explanar exaustivamente
todos
os passos presentes nas entrelinhas das demonstrações mais
avançadas. Recomendamos fortemente a consulta às referências
bibliográficas expostas para que o leitor possa contemplar melhor esses
conceitos fundamentais sobre polinômios e corpos finitos e facilmente
perceber que o que foi exposto aqui se trata de uma pequeníssima porção
de um universo fascinante.

No capítulo\xspace\ref{capalg}, por sua vez, nos aplicaremos finalmente
a exibir
o algoritmo. Discorreremos também um pouco, ainda que não muito
formalmente, sobre o custo do algoritmo e daremos nossas conclusões a
respeito.

Dado o contexto de nossa revisão bibliográfica, fornecemos também um
breve apêndice sobre Álgebra Moderna, abordando alguns dos conceitos e
algumas das demonstrações utilizadas. Para o leitor mais interessado no
assunto, indicamos também referências bibliográficas para o tema.

Para facilitar a leitura, disponibilizamos os seguintes recursos:
\begin{enumerate}
  \item uma lista de símbolos, contendo as principais notações adotadas
  no texto;
  \item uma numeração única para definições, notações, nomenclaturas e
  ob\-ser\-va\-ções, referenciadas pelo número do capítulo em que se
  encontram;
  \item uma numeração única para teoremas, propriedades, lemas e
  corolários, também referenciados pelo número do capítulo em que se
  encontram;
  \item um índice remissivo contemplando os principais conceitos
  utilizados nos capítulos e apêndices;
  \item notas de margem que ajudam na localização dos conceitos do
  índice remissivo.
\end{enumerate}

Por fim, ressaltamos que, embora o objetivo do presente trabalho não
seja fornecer uma base teórica completa sobre o assuto de
irredutibilidade polinomial em corpos finitos, acreditamos que ele possa
servir como uma apresentação do tema para alunos de graduação
e uma orientação bibliográfica para
pesquisas futuras. O algoritmo apresentado ainda vem
acompanhado de sugestões de implementação que podem ser experimentadas
por qualquer leitor com noções básicas de programação, e as
demonstrações mais complicadas podem ser melhor acompanhadas no material
indicado nas referências.

%%%%%%%%%%%%%%%%%%%%%%%%%%%%%%%%%%%%%%%%%%%%%%%%%%%%%%%%%%%%%%%%%%%%%%%%
% introducao.tex                                                       %
%                                                       fim do arquivo %
%%%%%%%%%%%%%%%%%%%%%%%%%%%%%%%%%%%%%%%%%%%%%%%%%%%%%%%%%%%%%%%%%%%%%%%%
