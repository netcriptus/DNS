%%%%%%%%%%%%%%%%%%%%%%%%%%%%%%%%%%%%%%%%%%%%%%%%%%%%%%%%%%%%%%%%%%%%%%%%
% fund.tex                                                             %
%                                                    início do arquivo %
%%%%%%%%%%%%%%%%%%%%%%%%%%%%%%%%%%%%%%%%%%%%%%%%%%%%%%%%%%%%%%%%%%%%%%%%

\chapter{Esboço dos fundamentos teóricos}\label{capfund}

Com o objetivo de fundamentar teoricamente o algoritmo exibido no
capítulo\xspace\ref{capalg}, esboçamos no presente capítulo a
conceituação elementar da Teoria de Polinômios
(seção\xspace\ref{polconceituacao}) e da Teoria de Anéis de Polinômios
  (seção\xspace\ref{aneispolinomios}) e, por último, esboçamos os
  conceitos de divisibilidade e irredutibilidade polinomial, chegando ao
  teorema\xspace\ref{maint}, de cuja aplicação extrairemos nosso
  algoritmo.

Como já mencionado na introdução do documento, daremos uma ênfase
especial aos conceitos e às demonstrações que poderão nos ser úteis
posteriormente e não contemplaremos nem mesmo os resultados mais
interessantes ou de maior impacto relacionados ao assunto. Além do mais,
deixaremos de abordar explicitamente
muitos conceitos e demonstrações
que eram estritamente relacionados ao nosso texto, optando por não
carregar demais o trabalho. Novamente sugerimos a consulta à
bibliografia para que o leitor acompanhe melhor esses conceitos e essas
demonstrações.

\section{Polinômios}\label{polconceituacao}

Atualmente, a Matemática Moderna define o conceito de polinômio graças
ao conceito de ``função com suporte finito'', elaborado especificamente
para a a\-xi\-o\-ma\-ti\-za\-ção da Teoria de Polinômios. O leitor
notará também que
toda a abordagem sobre polinômios que construiremos será, pelo menos em
primeira instância, bastante genérica, diferente daquela da Matemática
Clássica. Na realidade, todo o presente capítulo caberia muito bem no
apêndice sobre fundamentação algébrica, o qual preferimos destinar a
conceitos mais elementares ainda, como as estruturas algébricas
tradicionais da Álgebra Moderna.

\begin{Def}
  Sendo $(G,\ast)$ um grupo qualquer,
  dizemos que uma função qualquer $a$ de contradomínio $G$
  possui \Conceito{suporte finito}{suporte finito de uma
  função} em $(G,\ast)$ se e somente se o
  conjunto $a^{-1}(G\setminus \cj{e})$
  é finito, sendo $e$ o elemento neutro do grupo.
\end{Def}

\begin{Obs}
  Sendo $(R,+,\cdot)$ um anel qualquer e $a$ uma função com suporte
  finito em $(R,+)$,
  note-se que a soma
  $\sum_{k\in\MMN}a(k)$ é bem definida, já que é finito seu número de
  termos diferentes de $\zero$, e que ou $a^{-1}(R\setminus\cj{\zero})$
  é vazio ou possui um máximo.
\end{Obs}

Entendido o conceito de função com suporte finito, podemos nos arriscar
a uma definição de ``polinômio''.

\begin{Def}\label{nompolinomio}
  Sendo $(R,+,\cdot)$ um anel qualquer e $\funcao{a}{\MMN}{R}$ uma
  função com suporte
  finito em $(R,+)$, chamamos de \conceito{polinômio} sobre
  $(R,+,\cdot)$ com \Conceito{indeterminada}{indeterminada de um
  polinômio} $x$
  a expressão
  \begin{equation*}
    f(x) = \sum_{k\in\MMN}a(k)x^k
  \end{equation*}
  para qualquer $x\in R$. Ademais, se o conjunto
  $a^{-1}(R\setminus\cj{0}$, além de finito,
  também não
  for vazio, chamaremos de
  \Conceito{grau}{grau de um polinômio} de $f$
  o
  número natural
  \begin{equation*}
    \simb[grau do polinômio $f$]{d(f)} =
    \max{\bigl(a^{-1}(R\setminus\cj{\zero})\bigr)}\MMp
  \end{equation*}
  Se, entretanto, $a^{-1}(R\setminus\cj{zero}$ for vazio, convencionamos
  que $d(f)=+\infty$.
\end{Def}

\begin{Obs}
  Note-se que $f$ pode ser escrito na forma
  \begin{equation}\label{fpadraop}
    f(x) = \sum_{k=0}^{d(f)} a_kx^k\MMv
  \end{equation}
  denotando-se $a(k)$ por $\simb[$a(k)$, sendo $a$ uma função com
  suporte finito]{a_k}$.
  Assim, dizemos que
  a equação\xspace\ref{fpadraop} é a
  \Conceito{forma padrão}{forma padrão de um polinômio} de $f$. Ademais,
  note-se também que não apenas uma função com suporte finito determina
  um e só um polinômio, mas também que um polinômio determina uma e só
  uma função com suporte finito, motivo pelo qual nos permitimos a nos
  referir a um polinômio sem necessariamente explicitar sua função com
  suporte finito.
\end{Obs}

\begin{Nom}
  Sendo $f(x)$ um polinômio sobre um anel $(R,+,\cdot)$ e sendo $a$ sua
  função com suporte finito,
  chamamos $a_0$ de \Conceito{termo constante}{termo constante de um
  polinômio} de $f(x)$. Se $f(x)$ possui grau finito então também
  chamamos $a_{d(f)}$ de \Conceito{termo dominante}{termo dominante de
  um polinômio} de $f(x)$.
\end{Nom}

\begin{Nom}
  Um polinômio é dito \Conceito{constante}{polinômio constante} se e
  só se seu grau é $0$.
\end{Nom}

A nomenclatura\xspace\ref{nompip}, que apresentamos a seguir, trata de
um dos princípios mais importantes e elementares da Teoria de Polinômios
pura: o princípio da identidade de polinômios, apresentado por Descartes
no século XVII. O leitor mais atento deve perceber que não há confusão
entre os conceitos de identidade e de igualdade de polinômios. Como um
polinômio é uma expressão,
dois polinômios $f(x)$ e $g(x)$ sobre um anel
são iguais se $f(r)$ e $g(r)$ são iguais para todo elemento $r$ do
anel. Isso não significa necessariamente que as expressões que formam
$f$ e $g$ sejam equivalentes. Na verdade, um resultado muito conhecido
dentro da Teoria de Polinômios é que polinômios sobre um domínio de
integridade --- e, portanto, sobre um corpo ---
são idênticos se e somente se são iguais.

\begin{Nom}[Princípio da identidade de polinômios]\label{nompip}
  \index{princípio da identidade de polinômios}
  \mbox{}\marginpar%
  [\raggedright\hspace{0pt}\footnotesize %
    princípio da identidade de polinômios\\\vspace{\baselineskip}]%
  {\raggedleft\hspace{0pt}\footnotesize %
    princípio da identidade de polinômios\\\vspace{\baselineskip}}
  Dizemos que dois polinômios $f$ e $g$ sobre um anel $(R,+,\cdot)$ são
  \Conceito{idênticos}{polinômios idênticos}, e escrevemos
  $\simb[identidade entre os polinômios $f$ e $g$]{f\equiv g}$,
  se e só se $a_k=b_k$ para todo $k\in\MMN$, sendo $a$ a função com
  suporte finito do polinômio $f$ e $b$ a função com suporte finito do
  polinômio $g$.
\end{Nom}

\begin{Not}\label{notzeroumid}
  Sendo $(R,+,\cdot)$ um anel, usamos
  $\simb[polinômio $\zero(x)=\zero$]{\zero(x)}$,
  $\simb[polinômio $\um(x) = \um$]{\um(x)}$ e
  $\simb[polinômio identidade: $\id(x) = x$]{\id(x)}$
  para denotar respectivamente os polinômios:
  \begin{equation*}
    \begin{aligned}
      \zero(x) &= \zero\MMpv\\
      \um(x) &= \um\MMpv\\
      \id(x) &= x\MMp
    \end{aligned}
  \end{equation*}
  Costumamos também chamar $\id(x)$ de \conceito{polinômio identidade}.
\end{Not}

\begin{Propr}\label{proprpneqzero}
  Sendo $f$ um polinômio sobre um anel $(R,+,\cdot)$, se
  $f(x)\neq\zero(x)$
  então $a_{d(f)}\neq\zero$.
\end{Propr}

\begin{dem}
  É imediato que se $f(x)\equiv\zero(x)$ então $f(x)=\zero(x)$, já que,
  da notação\xspace\ref{notzeroumid} e do teorema\xspace\ref{nompip},
  para todo $r\in R$,
  \begin{equation*}
    f(r) = \zero\MMp
  \end{equation*}
  Portanto, se $f(x)\neq\zero(x)$ então $f(x)\nequiv\zero(x)$ e, assim,
  $a^{-1}(R\setminus\cj{\zero})\neq\emptyset$ e, conseqüentemente, pela
  definição de grau, como
  $d(f)=\max{\bigl(a^{-1}(R\setminus\cj{\zero})\bigr)}$,
  $a_{d(f)}\neq\zero$.
\end{dem}

Não podemos esquecer que uma das principais causas das transformações
profundas pelas quais a Álgebra passou no século XIX devem-se sobretudo
ao problema de encontrar raízes de uma equação. Se observarmos
atentamente, perceberemos que as equações estudadas na época eram na
realidade polinômios sobre o anel --- mais especificamente corpo --- dos
números reais. Assim, apresentamos a seguir o conceito genérico
atualmente aceito como o de raiz de um polinômio.

\begin{Def}\label{nomraiz}
  Sendo $f$ um polinômio sobre um anel $(R,+,\cdot)$,
  dizemos que um elemento $u$
  de $R$ é \Conceito{raiz}{raiz de um polinômio} de $f$ se e só se
  $f(u)=\zero$. Se, porém,
  todos os elementos de $R$ são raízes de $f$ então
  dizemos que $f$ é \Conceito{identicamente nulo}{polinômio
  identicamente nulo}, já que é igual ao polinômio $\zero(x)$.
\end{Def}

\begin{Not}
  Sendo $f$ um polinômio sobre um anel $(R,+,\cdot)$,
  $\simb[conjunto das raízes do polinômio $f$]{\rho(f)}$
  denota o conjunto das raízes de $f$.
\end{Not}

\begin{Not}
  Sendo $f$ um polinômio sobre um anel $(R,+,\cdot)$, usamos
  $\simb[polinômio $(-f)(x) = -\bigl(f(x)\bigr)$]{-f(x)}$ para
  denotar o polinômio
  \begin{equation*}
    (-f)(x) = -\bigl(f(x)\bigr)\MMp
  \end{equation*}
\end{Not}

\section{Anéis de polinômios}\label{aneispolinomios}

Em 1824, N. H. Abel provou que não há fórmula geral por radicais para
resolver equações de grau no mínimo $5$. Entretanto, todos sabiam que
alguns casos particulares dessas equações eram resolúveis por
radicais. O que, então, caracterizava essas equações especiais? Foi
respondendo a essa pergunta que Galois delineou pela primeira vez o
conceito de grupo, associando a cada equação um grupo de permutações das
raízes da equação e mostrando que a resolubilidade por radicais dependia
de uma propriedade de que esses grupos poderiam ou não gozar.

Assim,
podemos observar a relação estreita entre os polinômios e as estruturas
algébricas modernas já no século XIX. Não demorou muito para que os
polinômios fossem formalizados sobre anéis --- como já abordamos na
seção\xspace\ref{polconceituacao} --- e menos tempo ainda se passou para
que surgisse o conceito de anéis de polinômios, que abordaremos a
seguir.

\begin{Teo}\label{teoanelpoli}
  Sendo $(R,+,\cdot)$ um anel qualquer,
  $\simb[conjunto de todos os polinômios sobre $R$]{R[x]}$
  o conjunto de todos os polinômios sobre $R$ e as
  operações de adição e multiplicação de polinômios definidas como
  \begin{equation*}
    \begin{aligned}
      (f+g)(x) &= f(x)+g(x)\qquad\text{e}\\
      (fg)(x) &= f(x)g(x)\MMv
    \end{aligned}
  \end{equation*}
  $(R[x],+,\cdot)$ é um anel, o \conceito{anel de polinômios} sobre o
  anel $(R,+,\cdot)$.
\end{Teo}

\begin{dem}
  Tomemos $f_1$, $f_2$ e $f_3$ polinômios de $R[x]$ cujas formas padrões
  sejam:
  \begin{equation}\label{eqformaspadroes}
    \begin{aligned}
      f_1(x) &= \sum_{k=0}^{d(f_1)} a_kx^k\MMpv\\
      f_2(x) &= \sum_{k=0}^{d(f_2)} b_kx^k\MMpv\\
      f_3(x) &= \sum_{k=0}^{d(f_3)} c_kx^k\MMp\\
    \end{aligned}
  \end{equation}
  De (\ref{eqformaspadroes}), a associatividade e a comutatividade da
  adição de polinômios
  seguem da própria associatividade e comutatividade da adição de
  elementos de $R$. Verifica-se também a existência de um polinômio que
  sirva de elemento neutro para a adição de polinômios na medida em que
  $f_1(x)+\zero(x)=f_1(x)$.
  Ademais, o polinômio $(-f_1)(x)$
  serve de
  simétrico de $f_1(x)$ em relação a $\zero(x)$.
  Assim, concluímos que $(R[x],+)$ se trata um
  grupo abeliano.

  Como a associatividade da multiplicação de polinômios segue da própria
  associatividade da multiplicação de elementos de $R$, resta-nos apenas
  mostrar a distributividade da multiplicação de polinômios sobre a
  adição de polinômios. Para tanto, notemos, de
  (\ref{eqformaspadroes}),
    que
  \begin{equation*}
    \begin{aligned}
      f_1(x)\bigl(f_2(x)+f_3(x)\bigr)
      &= \sum_{k=0}^{d(f_1)} a_kx^k \biggl(
      \sum_{k=0}^{d(f_2)} b_kx^k + \sum_{k=0}^{d(f_3)} c_kx^k
      \biggr)\\
      &= \biggl(\sum_{k=0}^{d(f_1)} a_kx^k\biggr)
      \biggl(\sum_{k=0}^{d(f_2)} b_kx^k\biggr) +
      \biggl(\sum_{k=0}^{d(f_1)} a_kx^k\biggr)
      \biggl(\sum_{k=0}^{d(f_3)} c_kx^k\biggr)\\
      &= \bigl(f_1(x)+f_2(x)\bigr)+f_3(x)\MMv
    \end{aligned}
  \end{equation*}
  como queríamos mostrar.
\end{dem}

As
propriedades\xspace\ref{proprpolicomutativo},
\xspace\ref{proprpoliunidade}
e \xspace\ref{proprpolidi} relacionam as propriedades dos anéis dos
polinômios com as propriedades dos anéis sobre os quais aqueles
polinômios são formados. O leitor mais atento perceberá que não há uma
propriedade que defina como corpo o anel de polinômios sobre corpos. O
motivo é que tal propriedade não existe. Deixamos como um interessante
exercício a demonstração de que um anel de polinômios sobre corpos não é
necessariamente um corpo.

\begin{Propr}\label{proprpolicomutativo}
  Sendo $(R,+,\cdot)$ um anel, $(R,+,\cdot)$ é comutativo se e somente
  se $(R[x],+,\cdot)$ é comutativo.
\end{Propr}

\begin{dem}
  Admitamos inicialmente que $(R,+,\cdot)$ seja um anel
  comutativo. Como, do teorema\xspace\ref{teoanelpoli}, $(R,+,\cdot)$ é
  um anel, falta-nos apenas mostrar a comutatividade da multiplicação de
  polinômios, que segue imediatamente da comutatividade da multiplicação
  de elementos de $R$, já que $(R,+,\cdot)$ se trata de um anel
  comutativo.

  Por outro lado, suponhamos agora que $(R[x],+,\cdot)$ é que se trata
  de um anel comutativo. Já sabemos, por hipótese, que $(R,+,\cdot)$ é
  um anel. Assim, falta-nos, desta vez, apenas mostrar a comutatividade
  da multiplicação de elementos de $R$. Tomemos, então, $a$ e $b$ em
  $R$ e tomemos também os polinômios constantes
  \begin{equation*}
    f_a(x) = a\qquad\text{e}\qquad f_b(x) = b\MMp
  \end{equation*}
  Como $(R[x],,+,\cdot)$ é comutativo, $f_a\cdot f_b = f_b\cdot f_a$ e,
  portanto, $a\cdot b = b\cdot a$, e concluimos a demonstração.
\end{dem}

\begin{Propr}\label{proprpoliunidade}
  Sendo $(R,+,\cdot)$ um anel, $(R,+,\cdot)$ possui unidade se e somente
  se $(R[x],+,\cdot)$ possui unidade.
\end{Propr}

\begin{dem}
  Supondo que $(R,+,\cdot)$ possui unidade, é imediato verificar que
  $(R[x],+,\cdot)$, que é um anel por causa do
  teorema\xspace\ref{teoanelpoli},
  também possui unidade,
  já que, como $\um$ é a unidade de $(R,+,\cdot)$,
  \begin{equation*}
    \begin{aligned}
      \um(x)\cdot f(x) &= \um\biggl( \sum_{k=0}^{d(f)} a_kx^k \biggr)\\
                       &= \sum_{k=0}^{d(f)} 1\cdot a_k\cdot x^k \\
                       &= \sum_{k=0}^{d(f)} a_kx^k\\
                       &= f(x)\MMv
    \end{aligned}
  \end{equation*}
  para todo $f\in R[x]$.

  Por outro lado, suponhamos que $(R[x],+,\cdot)$ possua uma unidade
  \begin{equation*}
    u(x) = \sum_{k=0}^{d(u)}y_kx^k\MMp
  \end{equation*}
  Logo, para todo $a\in R$, sendo $f_a(x)$ o polinômio
  constante
  definido
  por $f_a(x)=a$, observa-se que $(f_a\cdot u)(x) = f_a(x)$ e que
  $(u\cdot f_a)(x) = f_a(x)$,
  já que $u$ é
  unidade do anel $(R[x],+,\cdot)$. Em particular, temos, para algum
  $x_0\in R$, que
  $(f_a\cdot u)(x_0) = f_a(x_0)$, $(u\cdot f_a)(x_0) = f_a(x_0)$
  e, portanto, que
  \begin{align*}
      a\biggl( \sum_{k=0}^{d(u)}y_k{x_0}^k \biggr) &= a \\
      \intertext{e que}
      \biggl( \sum_{k=0}^{d(u)}y_k{x_0}^k \biggr)a &= a\MMp
  \end{align*}
  Logo, como $\sum_{k=0}^{d(u)}y_k{x_0}^k\in R$, encontramos uma unidade
  para o anel $(R,+,\cdot)$, já que $\sum_{k=0}^{d(u)}y_k{x_0}^k$ vale
  como elemento neutro da multiplicação para qualquer elemento $a$ de
  $R$.
\end{dem}

\begin{Lem}\label{lemgraumais}
  Sendo $f_1$, $f_2$ e $f_3$ polinômios sobre um anel $(R,+,\cdot)$, se
  $f_1 \equiv f_2+f_3$ e se $f_2\nequiv\zero$
  então
  \begin{equation*}
    d(f_2)\leq d(f_1)\MMp
  \end{equation*}
\end{Lem}

\begin{dem}
  Suponhamos que $d(f_2)>d(f_1)$. Sendo $a$ e
  $b$ as funções com suporte finito de, respectivamente, $f_1$ e
  $f_2$, como
  \begin{align*}
      f_2(x) &\equiv
        a_{d(f_2)}x^{d(f_2)}+\sum_{k=0}^{d(f_2)-1}a_kx^k\MMv\\
      \intertext{e, portanto,}
      f_1(x) &\equiv
        a_{d(f_2)}x^{d(f_2)}+\sum_{k=0}^{d(f_2)-1}a_kx^k+f_3(x)\MMv
  \end{align*}
  e como, da propriedade\xspace\ref{proprpneqzero},
  já que $f_2\nequiv\zero$,
  $d(f_2)\in b^{-1}(R\setminus\cj{\zero})$, $d(f_2)\in
  a^{-1}(R\setminus\cj{\zero})$, uma vez que
  $b^{-1}(R\setminus\cj{\zero})\subseteq
  a^{-1}(R\setminus\cj{\zero})$. Assim, porque
  \begin{equation*}
    d(f_1)=\max{\bigl(a^{-1}(R\setminus\cj{\zero})\bigr)}\MMv
  \end{equation*}
  e porque $d(f_2)>d(f_1)$, temos que
  \begin{equation*}
    d(f_2)>\max{\bigl(a^{-1}(R\setminus\cj{\zero})\bigr)}\MMv
  \end{equation*}
  o que é um absurdo, já que $d(f_2)\in a^{-1}(R\setminus\cj{\zero}$.
\end{dem}

\begin{Lem}\label{lemgrauvezes}
  Sendo $f$ e $g$ polinômios sobre um domínio de integridade
  $(R,+,\cdot)$,
  \begin{equation*}
    d(fg) = d(f)+d(g)\MMp
  \end{equation*}
\end{Lem}

\begin{dem}
  Se $f(x)\equiv\zero(x)$ ou $g(x)\equiv\zero(x)$, a desigualdade se
  verifica trivialmente na medida em que $d(f)+d(g)=+\infty$. Assumamos,
  portanto, que $f(x)\nequiv\zero(x)$ e que $g(x)\nequiv\zero(x)$.
  Sabemos, sendo $a$ e $b$ as funções com suporte finito de,
  respectivamente, $f$ e $g$, que
  \begin{equation*}
      f(x) \equiv a_{d(f)}x^{d(f)} + f_1(x)\qquad\text{e que}\qquad
      g(x) \equiv b_{d(g)}x^{d(g)} + g_1(x)\MMv
  \end{equation*}
  para algum polinômio $f_1$ e algum polinômio $g_1$ sobre
  $(R,+,\cdot)$. Assim,
  \begin{equation}\label{eqidlemgrauvezes}
    f(x)g(x) \equiv a_{d(f)}b_{d(g)}x^{d(f)+d(g)}
      + a_{d(f)}x^{d(f)}g_1(x) + b_{d(g)}x^{d(g)}f_1(x)\MMp
  \end{equation}
  Como $f(x)\neq \zero(x)$ e $g(x)\neq\zero(x)$, temos, da
  propriedade\xspace\ref{proprpneqzero}, que
  $a_{d(f)}\neq \zero$ e $b_{d(g)}\neq \zero$. Como $(R,+,\cdot)$ é um
  domínio de integridade, $a_{d(f)}b_{d(g)}\neq\zero$ e, portanto, do
  lema\xspace\ref{lemgraumais},
  \begin{equation*}
    d\bigl(a_{d(f)}b_{d(g)}x^{d(f)+d(g)}\bigr) = d(f)+d(g) \leq
    d\bigl(f(x)g(x)\bigr)\MMp
  \end{equation*}
  Resta-nos ainda provar que $d(fg)\leq d(f)+d(g)$. Suponhamos que
  $d(fg)>d(f)+d(g)$ e tomemos $c$ a função de suporte finito do
  polinômio $fg$, $\alpha$ a função com suporte finito de $f_1$ e
  $\beta$ a função com suporte finito de $g_1$. Tomemos também os
  conjuntos:
  \begin{equation*}
    \begin{aligned}
      A &= \left\{
      \begin{aligned}
        &\cjpp{d(g)+\ell}{\ell\in\alpha^{-1}(R\setminus\cj{\zero})}\MMv
        &\quad&\text{se $p_1(x)\nequiv \zero(x)$;}\\
        &\emptyset&\quad&\text{caso contrário;}
      \end{aligned}
      \right.\\
      B &= \left\{
      \begin{aligned}
        &\cjpp{d(f)+\ell}{\ell\in\beta^{-1}(R\setminus\cj{\zero})}\MMv
        &\quad&\text{se $q_1(x)\nequiv \zero(x)$;}\\
        &\emptyset&\quad&\text{caso contrário;}
      \end{aligned}
      \right.
    \end{aligned}
  \end{equation*}
  Como (\ref{eqidlemgrauvezes}) se trata de uma
  identidade polinomial, então
  \begin{equation*}
    c^{-1}(R\setminus\cj{\zero}) = \cj{d(f)+d(g)} \cup A\cup B\MMp
  \end{equation*}
  Contudo, $d(fg)\notin \cj{d(f)+d(g)}$, pois, por hipótese,
  $d(fg)>d(f)+d(g)$. Também $d(fg)\notin A$, pois, se $A$ não é vazio
  então $\max{A}=d(g)+d(f_1)$ e, portanto, $\max{A}\leq d(g)+d(f)$, uma
  vez que, do
  lema\xspace\ref{lemgraumais}, já que $f_1\nequiv\zero$
  ($A\neq\emptyset$), $d(f_1)\leq d(g)$. E, finalmente, $d(fg)\notin B$,
  pois, se $B$ não é vazio
  então $\max{B}=d(f)+d(g_1)$ e, portanto, $\max{B}\leq d(f)+d(g)$, uma
  vez que, do
  lema\xspace\ref{lemgraumais}, já que $g_1\nequiv\zero$
  ($B\neq\emptyset$), $d(g_1)\leq d(g)$. Assim, $d(fg)\notin
  c^{-1}(R\setminus\cj{\zero})$, o que é um absurdo, já que $c$ é a
  função de suporte finito de $pq$. Conseqüentemente, $d(fg)\leq
  d(f)+d(g)$.
\end{dem}

\begin{Propr}\label{proprpolidi}
  Sendo $(R,+,\cdot)$ um anel,
  $(R,+,\cdot)$ é um domínio de integridade se e só se $(R[x],+,\cdot)$
  é um domínio de integridade.
\end{Propr}

\begin{dem}
  Suponhamos inicialmente que $(R,+,\cdot)$ seja um domínio de
  integridade.
  Das propriedades\xspace\ref{proprpolicomutativo}
  e\xspace\ref{proprpoliunidade}, como $(R,+,\cdot)$ é um anel
  comutativo com unidade, verifica-se também que $(R[x],+,\cdot)$ é um
  anel comutativo com unidade.
  Para mostrarmos que vale a lei do anulamento do produto,
  sendo $f$ e $g$
  polinômios de $R[x]$,
  utilizaremos a forma contrapositiva. Suponhamos que $f(x)\neq
  \zero(x)$ e que $g(x)\neq\zero(x)$.
  Do lema\xspace\ref{lemgrauvezes}, como $(R,+,\cdot)$ é um domínio de
  integridade, $d(fg)=d(f)+d(g)$. Como $d(f)$ e
  $d(g)$ são números naturais finitos, $d(fg)$ também é um número
  natural finito. Portanto, $(fg)(x)\neq \zero(x)$, como queríamos
  mostrar.

  Admitamos agora que $(R[x],+,\cdot)$ é que se trata de um domínio de
  integridade. Por hipótese, $(R,+,\cdot)$ é um anel e, das
  propriedades\xspace\ref{proprpolicomutativo}
  e\xspace\ref{proprpoliunidade}, é comutativo e possui unidade. Por
  fim, para mostrarmos que vale a lei do anulamento do produto, também
  utilizaremos a forma contrapositiva. Notemos que, para $a$ e $b$ em
  $R\setminus\cj{\zero}$,
  vale, já que $(R[x],+,\cdot)$ se trata de um domínio de
  integridade,
  que $(f_a\cdot f_b)(x)\neq \zero(x)$, sendo $f_a$ e $f_b$ os
  polinômios constantes $f_a(x)=a$ e $f_b(x) = b$. Portanto,
  $ab\neq\zero$, como queríamos mostrar.
\end{dem}

\section{Divisibilidade e irredutibilidade polinomial}

Por último, apresentamos os conceitos aos quais pretendíamos chegar. O
leitor notará a relação direta entre o conceito de irredutibilidade
polinomial e o conceito de divisibilidade polinomial, motivo pelo qual
decidimos apresentar ambas as definições numa mesma seção. Vale lembrar
que é justamente pertinente a esses assuntos que a maioria dos
resultados mais interessantes conhecidos são deixados de lado, dada a
imensidão do que já se conhece.

\begin{Def}
  Dizemos que um polinômio $f_1$ de um anel de polinômios $(R[x],+,\cdot)$
  divide um outro polinômio $f_2$ do mesmo anel, e escrevemos
  $\simb[divisibilidade do polinômio $f_1$ pelo polinômio
    $f_2$]{f_1\divd f_2}$,
  se e só se existe polinômio $g\in R[x]$ tal que $f_2=f_1g$.
\end{Def}

\begin{Nom}\label{nomdivibddpoli}
  Quando $f_1\divd f_2$, sendo $f_1$ e $f_2$ polinômios de um anel de
  polinômios
  $(R[x],+,\cdot)$, dizemos que:
  \begin{enumerate}[(i)]
    \item $f_2$ é \Conceito{divisível}{divisibilidade polinomial}
      por $f_1$;
    \item $f_2$ é \Conceito{múltiplo}{múltiplo de um polinômio} de
      $f_1$;
    \item $f_1$ é \Conceito{divisor}{divisor de um polinômio} de $f_2$;
    \item $g$ é um \Conceito{quociente da divisão}{quociente de uma
      divisão polinomial} de $f_2$ por $f_1$,
      sendo $g$ um polinômio tal que $f_2=f_1g$.
  \end{enumerate}
\end{Nom}

O leitor deve ter notado a semelhança da
nomenclatura\xspace\ref{nomdivibddpoli} com aquela utilizada para
divisibilidade de números inteiros, que, a saber, é apresentada na
nomenclatura\xspace\ref{nomdivinteiro}. Não poderia ser de outro modo,
já que as construções teóricas são bastante similares. Assim, embora não
demonstraremos o resultado apresentado na
observação\xspace\ref{obsalgdiv}, não é de se assustar que ele valha e
que sua demonstração seja muito parecida com aquela utilizada para
mostrar a validade do algoritmo da divisão para números inteiros. O
leitor ainda notará que abandonaremos os polinômios sobre anéis
quaisquer e nos ateremos em especial aos anéis sobre corpos. Isso se faz
necessário para a construção do embasamento teórico que se segue.

\begin{Obs}[Algoritmo da divisão para polinômios]\label{obsalgdiv}
  \index{algoritmo da divisão para polinômios}
  \mbox{}\marginpar%
  [\raggedright\hspace{0pt}\footnotesize %
    algoritmo da divisão para polinômios\\\vspace{\baselineskip}]%
  {\raggedleft\hspace{0pt}\footnotesize %
    algoritmo da divisão para polinômios\\\vspace{\baselineskip}}
  Se $g(x)\neq\zero(x)$ é um polinômio de $F[x]$, sendo $(F,+,\cdot)$ um
  corpo, então para todo $f\in F[x]$ existem dois polinômios $q$ e $r$
  em $F[x]$ tais que
  \begin{equation*}
    f = qg + r\qquad\text{e}\qquad d(r)<d(g)\MMp
  \end{equation*}
\end{Obs}

O teorema\xspace\ref{teoraizdivide}, que apresentamos a seguir, é um dos
mais elementares da Teoria de Polinômios. Perceber-se-á que boa parte do
que for explanado neste texto a partir daqui será relacionado ao
resultado do teorema.

\begin{Teo}\label{teoraizdivide}
  Sendo
  $f$ um polinômio sobre um corpo $(F,+,\cdot)$,
  um elemento $\alpha\in F$ é uma raiz de
  $f$ se e só se $(x-\alpha)$ divide $f(x)$.
\end{Teo}

\begin{dem}
  Suponhamos inicialmente que $\alpha$ seja raiz de $F$.
  Como $x-\alpha$ não é $\zero$ para todo $x\in F$, temos, da
  observação\xspace\ref{obsalgdiv}, que $f(x)$ pode ser escrito como:
  \begin{equation*}
    f(x) = q(x)(x-\alpha)+r(x)\MMv
  \end{equation*}
  sendo $r(x)$ um polinômio de grau menor que o grau de $x-\alpha$,
  portanto, $0$. Assim,
  \begin{equation*}
    f(x) = q(x)(x-\alpha)+\beta\MMv
  \end{equation*}
  para algum $\beta\in F$. Dessarte, temos que
  \begin{equation*}
    f(\alpha) = q(\alpha)(\alpha-\alpha)+\beta
    = q(\alpha)(\zero)+\beta = \beta
  \end{equation*}
  e, conseqüentemente, que
  \begin{equation*}
    f(x) = q(x)(x-\alpha)+f(\alpha)\MMp
  \end{equation*}
  Mas, como $f(\alpha)=\zero$, já que $\alpha$ é uma raiz de $f$,
  \begin{equation*}
    f(x) = q(x)(x-\alpha)\MMp
  \end{equation*}
  Logo, concluímos que $x-\alpha$ divide $f(x)$.

  Por outro lado, admitamos agora que $x-\alpha$ divide $f(x)$. Assim,
  $f(x)$ pode ser escrito como:
  \begin{equation*}
    f(x) = (x-\alpha)q(x)\MMp
  \end{equation*}
  Portanto,
  \begin{equation*}
    f(\alpha) = (\alpha-\alpha)q(x) = (\zero)q(x) = \zero\MMv
  \end{equation*}
  como queríamos demonstrar.
\end{dem}

A seguir, chegamos na definição que contribui para o título do presente
trabalho. O conceito de irredutibilidade polinomial sobre corpos,
como se nota,
se trata na verdade de um conceito bastante simples, mas muito
poderoso. Futuramente comentaremos um pouco da importância de se testar
a irredutibilidade de polinômios sobre corpos finitos e de se
encontrarem polinômios irredutíveis sobre corpos finitos.

\begin{Def}
  Seja $(F,+,\cdot)$ um corpo.
  Dizemos que um polinômio $p\in F[x]$ é
  \Conceito{irredutível}{polinômio irredutível} sobre $F[x]$ (ou
  irredutível em $F[x]$, ou
  \Conceito{primo}{polinômio primo} em $F[x]$)
  quando e só quando $d(p)>0$ e, para quaisquer $p_1$ e $p_2$ em
  $F[x]$,
  $p=p_1p_2$
  implicar sempre que ou $p_1$ ou $p_2$
  seja um polinômio constante.
\end{Def}

\begin{Not}
  Sendo $K$ um subcorpo de um corpo
  $(F,+,\cdot)$ e $M$ qualquer subconjunto de $F$, usamos
  $\simb[o menor subcorpo que contém
  ambos $K$ e $M$]{K(M)}$ para
  denotar o conjunto
  \begin{equation*}
    K(M) = \bigcap_{L\subseteq (F,+,\cdot)\\K\cup M\subseteq L}L\MMp
  \end{equation*}
\end{Not}

\begin{Propr}
  Sendo $K$ um subcorpo de um corpo
  $(F,+,\cdot)$ e $M$ qualquer subconjunto de $F$, $K(M)$ é um subcorpo
  de $(F,+,\cdot)$.
\end{Propr}

\begin{dem}
  Sabemos que $\zero\in K(M)$, já que,
  do teorema\xspace\ref{teosubcorpo},
  $\zero$ pertence a qualquer
  subcorpo $L$ de $(F,+,\cdot)$. Tomemos agora $x$ e $y$ elementos de
  $K(M)$. Assim, $x$ e $y$ são elementos de qualquer subcorpo $L$ de
  $(F,+,\cdot)$ tal que $K\cup M\subseteq L$. Suponhamos, então, que
  $x-y\notin K(M)$. Logo, existe um subcorpo $L_0$ de
  $(F,+,\cdot)$ tal que $K\cup M\subseteq L_0$ e tal que $x-y\notin
  L_0$, o que, também por causa do teorema\xspace{teosubcorpo},
  é um absurdo, já que $L_0$ é um subcorpo e já que $x$ e
  $y$ são elementos de $L_0$.

  Vamos agora mostrar que
  se $x\in K(M)$ e $y\in K(M)\setminus\cj{\zero}$ então $xy^{-1}\in
      K(M)$. Se $K(M)\setminus\cj{\zero}=\emptyset$ então a condicional
  se verifica trivialmente. Portanto, assumamos que
  $K(M)\setminus\cj{\zero}\neq\emptyset$ e tomemos $x\in K(M)$ e
  $y\in K(M)\setminus\cj{\zero}$. Dessarte, $x$ e $y$ são elementos de
  qualquer subcorpo $L$ de
  $(F,+,\cdot)$ tal que $K\cup M\subseteq L$.
  Se, contudo, supuséssemos que $xy^{-1}\notin K(M)$,
  teríamos a existência de um subcorpo $L_0$ de $(F,+,\cdot)$
  tal que $K\cup M\subseteq L$ e tal que $xy^{-1}\notin L_0$, o que
  seria um absurdo, dado que, por $L_0$ ser um subcorpo e por $x$ e $y$
  pertencerem a $L_0$, $L_0$ deveria conter $xy^{-1}$ por causa do
  teorema\xspace\ref{teosubcorpo}.

  Finalmente, por causa do teorema\xspace\ref{teosubcorpo},
  temos que $K(M)$ é um subcorpo de $(F,+,\cdot)$.
\end{dem}

O conceito que apresentamos a seguir, de ``corpos de decomposição'', é
de fato bastante relacionado com o conceito de ``corpos de extensão'',
que não abordamos --- o que não significa que não utilizamos ---
com mais detalhes no presente
trabalho. Em linhas gerais, um se $K$ é um subcorpo de um corpo
$(F,+,\cdot)$, dizemos que $(F,+,\cdot)$ é um \conceito{corpo de
  extensão} de $K$. O conceito pode ser facilmente entendido quando
pensamos no corpo dos números complexos como um corpo de
extensão do corpo dos números reais. Na verdade, a principal motivação
para o estudo de corpos de extensão está justamente no estudo de raízes
de polinômios sobre corpos. Sabemos que nem todas as raízes de
polinômios sobre o corpo dos reais pertencem a $\MMR$, mas é muito
conhecido que todas elas pertencem a $\MMC$. Uma argumentação, que
não será devidamente abordada no presente trabalho, mostra um resultado
parecido para corpos finitos. Em particular, todas as raízes de um
polinômio de grau $n$ sobre $\galois{p}$ estão no corpo de extensão
$\galois{p^n}$.

\begin{Def}
  Seja $K$ um subcorpo de um corpo
  $(F,+,\cdot)$.
  Dizemos que um polinômio $p\in K[x]$ de grau positivo finito
  \Conceito{decompõe}{decomposição de um corpo por um
  polinômio}\footnote{Em inglês,
  \textit{splits}.}
  $(F,+,\cdot)$ se e somente se existe um subconjunto finito
  $A=\cjpp{\alpha_j}{j\in[\cardi{A}]}$ de $F$ tal que
  \begin{equation*}
    p(x) = a_0\prod_{j=1}^{\cardi{A}}(x-\alpha_j)\MMv
  \end{equation*}
  sendo $a_0$ o termo constante de $p(x)$. Ademais, dizemos que
  $(F,+,\cdot)$ é um \conceito{corpo de decomposição
  polinomial}\footnote{Em inglês, \textit{splitting
  field}\index{\textit{splitting field}}.} de $p$ sobre $K$
  se e só se $p$ decompõe $(F,+,\cdot)$ e $F=K(A)$.
\end{Def}

\begin{Lem}
  Sendi $q$ uma potência de um primo, sendo $f\in \galois{q}[x]$ um
  polinômio irredutível sobre $\galois{q}$ de grau $m$ e sendo $n$ um
  número natural, $f(x)$ divide $x^{q^n}-x$ se e somente se $m$ divide
  $n$.
\end{Lem}

\begin{dem}
  Suponhamos inicialmente que $f(x)$ divida $x^{q^n}-x$ e tomemos
  $\alpha$ uma raiz de $f$ no corpo de decomposição de $f$ sobre
  $\galois{q}$. Como $\alpha$ é raiz de $f(x)$, temos do
  teorema\xspace\ref{teoraizdivide} que $x-\alpha$ divide $f(x)$ e,
  portanto, que $x-\alpha$ divide $x^{q^n}$. Assim, novamente do
  teorema\xspace\ref{teoraizdivide}, $\alpha$ é uma raiz de $x^{q^n}$ e,
  conseqüentemente, $\alpha=\alpha^{q^n}$. Logo, do teorema
  \xspace\ref{teoacardifa},
  $\alpha\in\galois{q^n}$, e, mais que isso, $\galois{q}(\cj{\alpha})$ é
  um subcorpo de $\galois{q^n}$. Entretanto, como
  $[\galois{q}(\cj{\alpha}):\galois{q}]=m$ e
  $[\galois{q^n}:\galois{q}]=n$, temos, do
  teorema\xspace\ref{teomkmllm}, que $m$ divide $n$.

  Assumamos agora que $m$ divida $n$. Da
  propriedade\xspace\ref{proprgaloismn}, $(\galois{q^m},+,\cdot)$
  é um subcorpo de $(\galois{q^n},+,\cdot)$. Tomemos agora uma raiz
  $\alpha$ de $f$ no corpo de decomposição de $f$ sobre
  $\galois{q}$. Logo, $[\galois{q}(\alpha):\galois{q}]=m$, e,
  conseqüentemente, $\galois{q}(\alpha)=\galois{q^m}$. Finalmente,
  como $\alpha\in\galois{q^n}$,
  $\alpha^{q^n}=\alpha$ (teorema\xspace\ref{teoacardifa}),
  o que nos leva a concluir que $\alpha$ é uma
  raiz também de $x^{q^n}-x$. Dessarte, $f(x)$ divide $x^{q^n}-x$, como
  queríamos mostrar.
\end{dem}

\begin{Teo}\label{everyroot}
  Sendo $f$ um polinômio irredutível de grau $m$ em $\galois{q}[x]$, $f$
  possui uma raiz $\alpha$ em $\galois{q^m}$. Ademais, são também raízes
  de $f$ todos os elementos do conjunto
  \begin{equation*}
    \cjpp{\alpha^{q^j}\in \galois{q^m}}{j\in[0..(m-1)]}\MMp
  \end{equation*}
\end{Teo}

\begin{dem}
  Notemos primeiramente que, tomando uma raiz $\alpha$ de $f$ no corpo
  de decomposição de $f$ sobre $\galois{q}$, temos que
  $[\galois{q}(\alpha):\galois{q}]=m$ e, portanto, que
  $\galois{q}(\alpha)=\galois{q^m}$. Em particular,
  $\alpha\in\galois{q^m}$. Ademais, se tivermos $\beta\in\galois{q^m}$
  uma raiz de $f$, teremos que, sendo $a$ uma função com suporte finito
  para $f$,
  \begin{equation*}
    \begin{aligned}
    f(\beta^q) &= \sum_{j=0}^m a_j\beta^{qj}\\
               &= \sum_{j=0}^m a_j^q\beta^{qj}
    \qquad\text{(teorema\xspace\ref{teoacardifa})}\\
               &= \biggl(\sum_{j=0}^m a_j\beta^j\biggr)^q
    \qquad\text{(observação\xspace\ref{obsabpnapnbpn})}\\
               &= f(\beta)^q \\ &= \zero\MMp
    \end{aligned}
  \end{equation*}
  Assim, provamos que sempre que $\beta\in\galois{q^m}$,
  $\beta^q$ é uma raiz de $f$ e, conseqüentemente,
  $\beta^q\in\galois{q^m}$. Como havíamos mostrado que
  $\alpha\in\galois{q^m}$, podemos, então, concluir que
  todos os elementos do conjunto
  \begin{equation*}
    \cjpp{\alpha^{q^j}\in \galois{q^m}}{j\in[0..(m-1)]}
  \end{equation*}
  são raízes de $f$, como queríamos mostrar.
\end{dem}

Por fim, apresentamos o teorema que será a base para a construção do
algoritmo que estudaremos no capítulo\xspace\ref{capalg}.

\begin{Teo}\label{maint}
  Sendo $n$ um natural não nulo, $p$ um primo positivo,
  \begin{equation*}
    L=\cjpp{\ell_j}{1\leq j\leq n}
  \end{equation*}
  o
  conjunto de todos os divisores primos de $n$, $k=|L|$ e
  \begin{equation*}
    m_j=\frac{n}{\ell_j}\MMv\quad\forall j\in[k]\MMv
  \end{equation*}
  um polinômio $g\in\MMZ_p[x]$ de grau
  $n$ é irredutível em $\MMZ_p[x]$ se e só se:
  \begin{enumerate}[({\ref{maint}}.i)]
    \item\label{mainti} $g(x)\divd \bigl(x^{p^n} -x\bigr)$;
    \item\label{maintii} para todo $j\in[k]$, o polinômio constante
      $\um(x)$
    é o único polinômio que divide ambos $g(x)$ e $x^{p^{m_j}}-x$.
  \end{enumerate}
\end{Teo}

\begin{prova}
  Assumamos inicialmente que $g(x)$ seja irredutível em $\MMZ_p[x]$.
  Do
  teorema\xspace\ref{everyroot}, temos que toda raiz $\alpha$ de $g(x)$
  pertence a
  $\galois{p^n}$. Como $\galois{p^n}$ é um corpo finito com $p^n$
  elementos, temos, do teorema\xspace\ref{teoacardifa}, que
  \begin{equation*}
    \alpha^{p^n} = \alpha\MMv
  \end{equation*}
  o que caracteriza $\alpha$ como raiz do polinômio $x^{p^n}-x$ e
  nos traz, do teorema\xspace\ref{teoraizdivide}, que
  \begin{equation*}
    (x-\alpha)\divd \bigl(x^{p^n}-x\bigr)\MMp
  \end{equation*}
  Logo,
  \begin{equation*}
    g(x)\divd \bigl(x^{p^n} -x\bigr)
  \end{equation*}
  e provamos (\ref{mainti}). Ademais, temos que,
  para qualquer natural $m$
  menor que $n$, $g(x)$ não tem raízes em $\galois{p^m}$. Portanto, para
  toda raiz $\alpha$ de $g(x)$ e todo natural $m$ menor que $n$,
  sabemos,
  também do teorema\xspace\ref{teoraizdivide},
  que
  \begin{equation*}
    (x-\alpha)\ndivd \bigl(x^{p^m}-x\bigr)\MMp
  \end{equation*}
  Como para todo divisor $d(x)$ de $g(x)$ diferente do polinômio
  constante
  $\um(x)$
  existe um subconjunto $\Gamma$ do conjunto das raízes de $g(x)$
  tal que
  \begin{equation*}
    \prod_{\gamma\in\Gamma}(x-\gamma) = d(x)\MMv
  \end{equation*}
  temos que nenhum divisor de $g(x)$ diferente do polinômio constante
  $\um(x)$
  divide $x^{p^m}-x$, sendo $m$ um natural menor que $n$. Assim, em
  particular, temos que nenhum divisor de $g(x)$ diferente do polinômio
  constante $\um(x)$ divide $x^{p^m_j}-x$, para todo $j\in[k]$, e
  provamos (\ref{maintii}).

  Inversamente, assumamos agora (\ref{mainti}) e (\ref{maintii}). Como
  \begin{equation*}
    (x-\alpha)\divd \bigl(x^{p^n}-x\bigr)\MMv
  \end{equation*}
  qualquer que seja $\alpha$ raiz de $g(x)$, todas as raízes de $g(x)$
  estão em $\galois{p^n}$. Queremos demonstrar que $g$ é
  irredutível. Suponhamos, entretanto, que $g$ seja redutível e tomemos
  $g_1$ um divisor irredutível não constante
  de $g$. Sendo $m$ o grau de $g_1$, com
  $m<n$, sabemos, do teorema\xspace\ref{everyroot},
  que todas as raízes de $g_1(x)$ pertencem a $\galois{p^m}$, que é
  gerado sobre $\MMZ_p$ por qualquer uma dessas raízes,
  conforme o mesmo teorema\xspace\ref{everyroot}.
  Já que $g_1$ é um divisor de $g$, temos que $\galois{p^m}\subseteq
  \galois{p^n}$ e que $m\divd n$. Como $m<n$, é trivial que $m\divd
  m_t$ para algum $t\in [k]$. Assim, analogamente, todas as
  raízes de $g_1$ estão em $\galois{p^{m_t}}$. Portanto, $g_1(x)$,
  que não é o polinômio constante $\um(x)$,
  divide ambos $g(x)$ e $x^{p^{m_t}}-x$,
  o que contraria
  (\ref{maintii}). Conseqüentemente, $g(x)$ é irredutível.
\end{prova}

%%%%%%%%%%%%%%%%%%%%%%%%%%%%%%%%%%%%%%%%%%%%%%%%%%%%%%%%%%%%%%%%%%%%%%%%
% fund.tex                                                             %
%                                                       fim do arquivo %
%%%%%%%%%%%%%%%%%%%%%%%%%%%%%%%%%%%%%%%%%%%%%%%%%%%%%%%%%%%%%%%%%%%%%%%%
