%%%%%%%%%%%%%%%%%%%%%%%%%%%%%%%%%%%%%%%%%%%%%%%%%%%%%%%%%%%%%%%%%%%%%%%%
% abstract.tex                                                         %
%                                                    início do arquivo %
%%%%%%%%%%%%%%%%%%%%%%%%%%%%%%%%%%%%%%%%%%%%%%%%%%%%%%%%%%%%%%%%%%%%%%%%


\chapter*{Abstract}

While Modern Math was being developed, many algebric structures were
proposed with the objective of organizing logically and axiomatizing the
Algebra. Among these structures, one which has been received special
attention from the mathematicians in last decades is the finite case of
the structure called ``field''. Basically, a field is a mathematic
system in which are available the addition, the subtraction, the
multiplication and the division for all the elements (excepting division
by zero). Fields are a particular case of rings. Although polynomials
can be constructed over any rings, the interest in the study of
finite fields comes associated to the interest in the study of
polynomials over finite fields. Polynomials over finite fields have many
applications in Math,
Engineering and Technology. Particularly, the study of polynomial
irreducibility over finite fields is very important. The present work,
therefore, presents an algorithm, proposed by  Michael O. Rabin, for
testing the irreducibility of a polynomial of degree $n$ over a finite
field with $p$ elements, analyzes
the algorithm, whose execution time is given by
\begin{equation*}
  T(n) = O\bigl((n\log{n}(\log{\log{n}})\log{p})^2\bigr)\MMv
\end{equation*}
 and also shows a sketch of the theoretical foundations in
the background of the used concepts, leaving a bibliographic orientation
and a motivation for future researches.

%%%%%%%%%%%%%%%%%%%%%%%%%%%%%%%%%%%%%%%%%%%%%%%%%%%%%%%%%%%%%%%%%%%%%%%%
% abstract.tex                                                         %
%                                                       fim de arquivo %
%%%%%%%%%%%%%%%%%%%%%%%%%%%%%%%%%%%%%%%%%%%%%%%%%%%%%%%%%%%%%%%%%%%%%%%%


