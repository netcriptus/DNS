%%%%%%%%%%%%%%%%%%%%%%%%%%%%%%%%%%%%%%%%%%%%%%%%%%%%%%%%%%%%%%%%%%%%%%%%
% resumo.tex                                                           %
%                                                    início do arquivo %
%%%%%%%%%%%%%%%%%%%%%%%%%%%%%%%%%%%%%%%%%%%%%%%%%%%%%%%%%%%%%%%%%%%%%%%%


\chapter*{Resumo}

No desenvolvimento da Matemática Moderna, várias estruturas algébricas
foram propostas com o objetivo de organizar logicamente e axiomatizar a
Álgebra. Dentre essas estruturas, uma que tem recebido atenção especial
dos matemáticos nas últimas décadas é o caso finito
da estrutura chamada de
``corpo''. Basicamente, um corpo é um sistema matemático em que valem a
adição, a subtração, a multiplicação e a divisão para todos os elementos
(com exceção da divisão por zero). Corpos são um caso particular de
anéis. Embora polinômios possam ser construídos sobre anéis quaisquer, o
interesse no estudo dos corpos finitos vem associado ao interesse no
estudo de polinômios sobre corpos finitos. Polinômios sobre corpos
finitos têm aplicações diversas na Matemática, na Engenharia e na
Tecnologia. Em particular, o estudo da irredutibilidade polinomial sobre
corpos finitos é muito importante. O presente trabalho, portanto,
apresenta um algoritmo, proposto por Michael O. Rabin,
para testar a irredutibilidade de um polinômio de grau $n$ sobre um
corpo finito com $p$ elementos,
 analisa o algoritmo, cujo tempo de execução é
dado por
\begin{equation*}
  T(n) = O\bigl((n\log{n}(\log{\log{n}})\log{p})^2\bigr)\MMv
\end{equation*}
e ainda fornece um
esboço da fundamentação teórica por trás dos conceitos empregados,
deixando uma orientação bibliográfica e uma motivação para pesquisas
futuras.

%%%%%%%%%%%%%%%%%%%%%%%%%%%%%%%%%%%%%%%%%%%%%%%%%%%%%%%%%%%%%%%%%%%%%%%%
% resumo.tex                                                           %
%                                                       fim de arquivo %
%%%%%%%%%%%%%%%%%%%%%%%%%%%%%%%%%%%%%%%%%%%%%%%%%%%%%%%%%%%%%%%%%%%%%%%%
