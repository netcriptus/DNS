\begin{thebibliography}{9}
\addcontentsline{toc}{chapter}{Bibliography}

% Infelizmente escrever bibitems não é lá muito fácil, embora seja
% mais fácil que usar o BIBTEX. Estes bibitems são exemplos de como
% você deve fazer para criar os itens de sua bibliografia.
% Lembre-se: eles aparecerão na ordem em que você os escrever.

% A estrutura de um bibitem é a seguinte:
% 1) o comando \bibitem
% 2) as strings que serão usadas para gerar a citação, ou seja [AUTOR(ANO)]
% 3) o nome que será usado para identificar o item.
% 4) o nome real do(s) autores, como deverá aparecer na Bibliografia.
% 5) o resto da citação.


% Citação de artigo em revista:
%\bibitem[ACOSTA(1999)]{acosta}ACOSTA, Maria A. G. Valim. Alunos Com
%Motivação Instrumental em um Curso de Inglês Geral: Um Conflito de
%Objetivos. \emph{Intercâmbio}, Vol.~VIII, 1999 (pp. 65-90).

%Citação de artigo na internet:
%\bibitem[AZEVEDO(2004)]{atica}AZEVEDO, Guila Eitelberg.
%\emph{Entendendo os Parâmetros Curriculares Nacionais}.  Disponível
%em <http://www.aticaeducacional.com.br/htdocs/pcn/> Acesso em 5 de
%setembro de 2004.

% Citação de Manual
%\bibitem[PCN-EF(1999)]{pcn-ef}
%BRASIL. Ministério da Educação e do Desporto. Secretaria de Educação
%Fundamental. Parâmetros curriculares nacionais: terceiro e quarto
%ciclos do ensino fundamental: língua estrangeira. Secretaria de
%Educação Fundamental.  Brasília: MEC/SEF, 1998. 120 p.

% Citação de Monografia (Tese)
%\bibitem[CASTRO(1999)]{castro}CASTRO, Solange T.~Ricardo de. \emph{A
%Linguagem e o Processo  de Construção  do Conhecimento: Subsídios
%Para a Formação do Professor de Inglês}. Tese (Doutorado Lingüística Aplicada.) Faculdade de Letras, Pontifícia Universidade
%Católica de São Paulo, São  Paulo, SP, 1999.

% Citação de Notícia (no caso na internet, mas para um jornal só muda a
% maneira de citar a fonte.
%\bibitem[MEC(2004)]{mecnews}
%d'ARCANCHY, Heloísa. MEC vai analisar parâmetros curriculares do ensino médio.
%Ministério da Educação. Notícias. 10 de setembro de 2004. Disponível em <http://www.mec.gov.br/acs/asp/noticias/noticiasId.asp?Id=6912>
%Acesso em 10 de setembro de 2004.

% Citação de Obra de referência
\bibitem[RFC1034(1987)]{rfc1034}MOCKAPETRIS, Paul -- RFC 1034, 1987.

\bibitem[RFC1035(1987)]{rfc1035}MOCKAPETRIS, Paul -- RFC 1034, 1987.

% Citação de Artigo em Coletânea ou capítulo de livro.
\bibitem[Albitz\&liu(2001)]{albitz}ALBITZ, Paul \& LIU, Cricket -- DNS
  and Bind, P.11 (Ed. 4).
\emph{Tópicos em Lingüística Aplicada: O Ensino de Línguas
Estrangeiras}. Florianópolis. Editora da UFSC, 1998. (pp. 211--236).

\bibitem[COMER(1998)]{comer} COMER, Douglas E., \emph{Interligação em
  rede com TCP/IP}, Editora Campus, Rio de Janeiro -- RJ, 1998.

% Citação de Ensaio Breve, avulso
%\bibitem[LOSITO(2003)]{losito}LOSITO, S.~M. \emph{A revisão da
%função da escola e a proposta dos Parâmetros Curriculares
%Nacionais}.  Disponível em:
%<http://www.bdt.org.br/ea/capacitacao/interface/anexo4.pdf>. Acesso
%em 4 de setembro de 2004.

% Citação de Website
%\bibitem[SCHÜTZ(2004)]{schutz}SCHÜTZ, Ricardo. \emph{English Made in
%Brazil}. Disponível em: <http://www.sk.com.br>.
%Acesso em Fevereiro de 2004.

\end{thebibliography}

% Descomente esta linha a seguir se for usar Glossário
%\printglossary
